\documentclass[titlepage,a4paper]{article}

\usepackage{bbm}
\usepackage{a4wide}
\usepackage[colorlinks=true,linkcolor=black,urlcolor=blue,bookmarksopen=true]{hyperref}
\usepackage{bookmark}
\usepackage{fancyhdr}
\usepackage[spanish]{babel}
\usepackage[utf8]{inputenc}
\usepackage[T1]{fontenc}
\usepackage{graphicx}
\usepackage{float}
\usepackage{tikz}
\usepackage{amsfonts}
\usepackage{amsmath}

\pagestyle{fancy} % Encabezado y pie de página
\fancyhf{}
\fancyhead[R]{Resumen Probabilidad y Estadistica B - FIUBA}
\renewcommand{\headrulewidth}{0.4pt}
\fancyfoot[C]{\thepage}
\renewcommand{\footrulewidth}{0.4pt}

\begin{document}
\begin{titlepage} % Carátula
	\hfill\includegraphics[width=6cm]{logofiuba.jpg}
    \centering
    \vspace{20px}
    \newline
    
    \Huge \textbf{Probabilidad y Estadistica B}
    \vskip2cm
    \Large Resumen Probabilidad y Estadistica B\\
    Segundo cuatrimestre de 2021 
    
\end{titlepage}

\tableofcontents % Índice general
\newpage

\section{Axiomas de Probabilidad}

Una probabilidad es una funcion de P que a cada evento A le hace corresponder un numero real P(A) con las siguientes propiedades:
\begin{enumerate}
    \item $0 \leq P(A) \leq 1$ 
    \item $P(\Omega)$ = 1 
    \item $ A \cap B = \emptyset \Longrightarrow P(A \cap B) = P(A) + P(B)  $
    \item $P(\bar{A}) = 1  - P(A)$
\end{enumerate}

\section{Experimentos con resultados equiprobables}
\subsection{Laplace}
Evento $A$ con $M$ elementos y $\Omega$ espacio finito de $N$ elementos:
\begin{equation*}
    P(A) = \frac{M}{N} = \frac{card(A)}{card(\Omega)}
\end{equation*}

\section{Conteo}
\subsection{Regla del producto}
Sirve para conjunto de pares ordenados entre dos conjuntos $A$ y $B$: (cada uno de A con cada uno de B)
\begin{equation*}
    A \times  B = \{(a,b) : a \in A, b \in B \} = card(A) \cdot card(B)
\end{equation*}

\subsection{Permutaciones}
Sirve para saber de cuantas formas se pueden ordenar $n$ elementos de un conjunto:
\begin{equation*}
    n! = 1 \times 2 \times ...\times n
\end{equation*}

\subsection{Variaciones}
Sirve para subconjuntos ordenados de $k$ elementos, pertenecientes a un conjunto de $n$ elementos, se representa como $(n)_{k}$:
\begin{equation*}
    (n)_{k} = n(n-1) ... (n-k+1) = \frac{n!}{(n-k)!}
\end{equation*}
\textbf{OBS}: se hace con el boton $nPr$

\subsection{Combinaciones}
Sirve para subconjuntos sin ordenar de $k$ elementos, pertenecientes a un conjunto de $n$ elementos, se representa como $n \choose k$:
\begin{equation*}
    {n \choose k} = \frac{n!}{k!(n-k)!}
\end{equation*}
\textbf{OBS}: se hace con el boton $nCr$

\subsection{Bolas y urnas}
Sirve para bolas indistinguibles y urnas:
\begin{equation*}
    \# CP = {B+(U-1) \choose B}
\end{equation*}
\textbf{OBS}: se hace con el boton $nPr$

\section{Teoremas sobre conjuntos de eventos}

\subsection{Teorema 1}
Sea $A(n)$ una sucesion de eventos tales que $A_{n} \subset A_{n+1} \forall n$ y $A = \bigcup_{i=0}^{\infty} A_{i}$ :
\begin{equation*}
    P(A) = \lim_{n \to \infty } P(A_{n})
\end{equation*}


\subsection{Teorema 2}
Sea $A(n)$ una sucesion de eventos tales que $A_{n+1} \subset A_{n} \forall n$ y $A = \bigcap_{i=0}^{\infty} A_{i}$ :
\begin{equation*}
    P(A) = \lim_{n \to \infty } P(A_{n})
\end{equation*}

\subsection{Teorema $\sigma$-aditividad}
Sea $A = \bigcup_{i=0}^{\infty} A_{i} \in \mathcal{A} $ con los eventos $A_{i}$ mutuamente excluyentes 2 a 2, entonces:
\begin{equation*}
    P(A) = P(\bigcup_{i=0}^{\infty} A_{i}) = \sum_{i=1}^{\infty}P(A_{i})
\end{equation*}

\section{Relaciones entre dos eventos: probabilidad condicional e independencia}

\subsection{Probabilidad condicional}
Es la probabilidad que un evento $A$ se de, sabiendo que ya se dio el evento $B$ ($A$ dado $B$):
\begin{equation*}
    P(A|B) = \frac{P(A \cap B)}{P(B)}
\end{equation*}

\subsubsection*{Propiedades de la probabilidad condicional}
\begin{enumerate}
    \item $0 \leq P(A|B) \leq 1$, $\forall A \in \mathcal{A}$
    \item $P(\Omega|B) = 1$
    \item Si $A \cap C = \emptyset \rightarrow P(A \cup C | B) = P(A|B) + P(C|B)$ 
    \item Si $P(B) > 0$ :\begin{itemize}
        \item $P(A \cap B) = P(A|B)\cdot P(B) = P(B|A) \cdot P(A)$
        \item $P(A \cap B \cap C) = P(A|B \cap C) \cdot P(A|B) \cdot P(C) =P(A|B \cap C) \cdot P(B\cap C) $
    \end{itemize}
\end{enumerate}

\subsubsection*{Teorema de la probabilidad total}
Dada una particion de $\Omega$ en $B_{1},B_{2},...,B_{n}$ eventos, dado un evento superpuesto $A$, la probabilidad de $A$ es:
\begin{equation*}
    P(A) = \sum_{i=1}^{n} P(A|B_{i}) \cdot P(B_{i})
\end{equation*}

\subsection{Independencia de eventos}
Dos eventos son independientes cuando:
\begin{equation*}
    P (A \cap B) = P(A) \cdot P(B)
\end{equation*}
Esto implica que hay la misma proporcion de $B$ en $A$ que en todo $\Omega$ y viceversa.

\subsubsection*{Propiedades de la independencia de eventos}
\begin{enumerate}
    \item Si $A$ y $B$ son independientes, tambien lo son $\bar{A}$ y $B$, $A$ y $\bar{B}$, $\bar{A}$ y $\bar{B}$
    \item $A_{1},A_{2},...,A_{n}$ son independientes sii para cada subconjunto de mas de dos elementos, la interseccion de los sucesos coincide con el producto de las probabilidades.
\end{enumerate}


\subsection{Teorema de Bayes}
Sean $B_{1},B_{2},...,B_{k}$ una particion de $\Omega$, $A$ un evento de probabilidad positiva:
\begin{equation*}
    P(B_{i}|A) = \frac{P(A|B_{i}) \cdot P(B_{i})}{\sum_{j=1}^{k} P(A|B_{i}) \cdot P(B_{i})}
\end{equation*}
Se deduce de la definicion de probabilidad condicional y el teorema de probabilidad total.

\section{Variables aleatorias}
Sea ($\Omega, \mathcal{A}, P$) un espacio de probabilidad y $ \text{X: }\Omega \rightarrow \mathbb{R}$  una funcion, diremos que $X$ es una variable aleatoria si
$X^{-1}(B) \in \mathcal{A}$. Se puede calcular probabilidad como:
\begin{equation*}
    P(X^{-1}(B)) = P(X \in B)
\end{equation*}

\subsection{Funcion de distribucion}
Sea ($\Omega, \mathcal{A}, P$) un espacio de probabilidad y $X$ una V.A., definimos su \underline{funcion de distribucion}:
\begin{equation*}
    F_{X} (x) = P(X \leq x) \hspace{16px}  \forall x \in \mathbb{R}
\end{equation*}
Esta funcion se encarga de acumular probabilidad desde $-\infty$ hasta $x$.\\
\textbf{OBS}: $P(A < X \leq B) = F_{X}(B) - F_{X}(A)$
\subsubsection*{Propiedades de la funcion de distribucion}
\begin{enumerate}
    \item $F_{X} \in [0,1] \hspace{8px} \forall x \in \mathbb{R}$.
    \item $F_{X}$ es monotona np decreciente.
    \item $F_{X}$ es continua a derecha.
    \item $\lim_{x \to {-\infty}} F_{X}(x) = 0$ y $\lim_{x \to {+\infty}} F_{X}(x) = 1$
\end{enumerate}

\subsection{Soporte de una V.A}
El soporte de X es:
\begin{equation*}
    S_{X} = \{ x \in \mathbb{R} : F_{X}(x) - F_{X}(x^{-}) \neq 0 \vee \frac{dF_{X}(x)}{dx} \neq 0 \}
\end{equation*}

\subsection{Variables aleatorias discretas}
La variable $X$ tiene una distribucion discreta si hay un conjunto $A \in \mathbb{R}$ finito o infinito numerable, tal que $P(X \in A) = 1$.\\
Sea para cada $x \in A:$ $p_{X}(x) = P(X = x)$, se verifica que si $B \subset \mathbb{R}$: 
\begin{equation*}
    P(X \in B) = \sum_{x \in B \cap A} p_{X}(x)
\end{equation*}
Y en particular:
\begin{equation*}
    \sum_{x \in A} p_{X}(x) = 1 
\end{equation*}
Y la funcion de distribucion es dado un $B = (-\infty,t]$ resulta:
\begin{equation*}
    P(X \in B) = P(X \leq t) = F_{X}(t) = \sum_{x \leq t} p_{X}(x)
\end{equation*}

\subsection{Variables aleatorias continuas}
Una variable aleatoria es continua si:
\begin{enumerate}
    \item 
    \begin{enumerate}
        \item El conjunto de valores posibles se compone de todos los numeros que hay en un solo intervalo o una union excluyente de estos.
        \item Ninguno de estos valores tiene un valor de probabilidad positivo $P(x=c) = 0 \hspace{8px} \forall c \in \mathbb{R}$  
    \end{enumerate}
    \item Se dice que $X$ es una variable continua si existe una funcion $f_{X}: \mathbb{R} \to \mathbb{R}$, llamada \underline{funcion de densidad}
    de probabilidad, que satisface las siguientes condiciones:
    \begin{enumerate}
        \item $f_{X} \geq 0 \hspace{8px} \forall x \in \mathbb{R}$
        \item $\int_{-\infty}^{\infty} f_{X}(x) dx = 1$
        \item Para cualquier $a$ y $b$ tales que $-\infty < a<b< +\infty$:
        \begin{equation*}
            P(a<X<b) = \int_{a}^{b} f_{X}(x) dx
        \end{equation*}
    \end{enumerate}
\end{enumerate}

\subsubsection*{Teorema}
Sea $F_{X} (x)$ una funcion de distribucion de una V.A.C. (admite derivada), luego:
\begin{equation*}
    f_{X}(x) = \frac{dF_{X}(x)}{dx}
\end{equation*}
\textbf{OBS}: La funcion de densidad solo existe para V.A.C.

\subsection{Eventos equivalentes}
Dos eventos son equivalentes si acumulan la misma probabilidad. Para V.A.D. significa que ambos eventos tiene la misma probabilidad.

\section{Modelos continuos: distribuciones notables}
\subsection{Distribucion Uniforme}
Supongamos una V.A.C. que toma todos los valores sobre un intervalo $[a,b]$. Si $f_{X}(x)$ esta dada por:
$$f_{X}(x)= \left\{ \begin{array}{lcc}
    \frac{1}{b-a} &   si  & a<X<b \\
    \\0 &  e.o.c 
    \end{array}
\right.$$
Se denota como $X \sim \mathcal{U} (a,b)$
\subsection{Distribucion exponencial}
Una variable aleatoria tiene una distribucion exponencial de parametro $\lambda > 0$ si su funcion de densidad esta dada por:
$$f_{X}(x)= \left\{ \begin{array}{lcc}
    \lambda e^{-\lambda x} &   si  & x>0 \\
    \\0 &  e.o.c 
    \end{array}
\right.$$

Y su funcion de distribucion es:
\begin{equation*}
    F_{X}(x)=P(X \leq x) = \left\{ \begin{array}{lcc}
        0 &   si  & x<0 \\
        \\ \int_{0}^{\alpha} \lambda e^{-\lambda t} dt = 1- e^{-\lambda \alpha} & &  e.o.c 
        \end{array}
    \right.
\end{equation*}
\subsubsection*{Propiedades de la exponencial}
\begin{enumerate}
    \item (PERDIDA DE MEMORIA) Si $X \sim \mathcal{E} (\lambda)$ entonces $P(X>t+s|X>t) = P(X>s)$  $\forall \hspace{8px} t,s  \in \mathbb{R}$.
    \item Si $X$ es una V.A.C. y $P(X>t+s|X>t) = P(X>s)$  $\forall \hspace{8px} t,s \in \mathbb{R}^{+}$ entonces existe $\lambda >0 $ tal que $X \sim \mathcal{E}(\lambda)$
\end{enumerate}

\subsection{Funcion de Riesgo (para V.A.C.)}
Dada la funcion intensidad de fallas $\lambda (t)$, la funcion de distribucion es:
\begin{equation*}
    F(t)= 1-e^{-\int_{0}^{\infty} \lambda(s)ds} \hspace{16px} \text{si  } t>0
\end{equation*}

\subsection{Distribucion Gamma}
Se dice una V.A tiene distribucion Gamma de parametros $\lambda$ y $k$ si su funcion de densidad es:
\begin{equation*}
    f_{X}(x)= \frac{\lambda^k}{\Gamma(k)}x^{k-1}e^{-\lambda x} \hspace{16px} \text{si  } \{x>0\}
\end{equation*}

\subsection{Distribucion normal estandar}
La V.A. X que toma los valores $-\infty < x < +\infty $ tiene una distribucion normal estandar si su funcion de densidad es:
\begin{equation*}
    f_{X}(x) = \frac{1}{\sqrt{2\pi}} e^{\frac{-x^{2}}{2}}
\end{equation*}
Para calcular probabilidades de esta distribucion hay que mirar la tabla o integrar numericamente.

\subsection{Cuantil de una V.A}

Un cuantil $\alpha$ de $X$ es cualquier numero $x_{\alpha}$ tal que :
\begin{equation*}
    P(X<x_{\alpha}) \leq \alpha \text{  y  } P(X>x_{\alpha}) \leq 1-\alpha
\end{equation*}

\section{Funciones de variables aleatorias}
Sea $ Y = g(X)$ con X una variable aleatoria:\\
Si $X$ es una V.A.D., Y sera discreta con:
\begin{equation*}
    p_{Y}(y) = P(Y=y) = \sum_{x \in B} p_{x}(x) \hspace{16px} \text{ con } \hspace{8px} B = \{ x\in \mathbb{R}: g(x)=y \}
\end{equation*}
Y en general:
\begin{equation*}
    F_{Y}(y)= P(Y \leq y) = P(g(x) \leq y)
\end{equation*}
Y con esta ultima se calcula la probabilidad $\forall y \in \mathbb{R}$

\subsection{Simulacion}
Sabiendo la distribucion de $X$ y teniendo una variable aleatoria $U$ para generar valores al azar, sabiendo su distribucion, entonces
se busca una $F_{U}(u_{i}) = F_{X}(x_{i})$, de donde se puede despejar $x_{i}$ en funcion de $u_{i}$.
Este despeje se puede hacer mediante la \underline{INVERSA GENERALIZADA}:
\begin{equation*}
    F_{X}^{-1}(u) = min \{x \in \mathbb{R}: F_{X}(x) \leq u \} \hspace{16px} \text{con } u \in (0,1)
\end{equation*}

\subsubsection*{Teorema}
Si f es una funcion que cumple:
\begin{itemize}
    \item Ser no decreciente.
    \item $\lim_{x \to +\infty}F(x) = 1$ y $\lim_{x \to -\infty}F(x) = 0$.
    \item Continua a derecha.
\end{itemize}
$\Rightarrow X = F^{-1}(U)$ con $U \sim \mathcal{U}(0,1)$, se tiene que $X$ es una V.A. cuya funcion de de distribucion es la funcion F dada.

\section{Truncamiento}
Sea $X$ una V.A con $F_{X}(x) = P(X\leq x)$
\begin{equation*}
    F_{X|X \in A}(x) = P(X \leq x | X \in A)\\
    \hspace{60px}=\frac{P(X \leq x, X \in A)}{P(X \in A)}
\end{equation*}
Si X es continua, $f_{X}(x) = \frac{dF_{X}(x)}{dx}$
\begin{equation*}
    \Rightarrow f_{X|X \in A}(x) = \frac{d}{dx}F_{X|X \in A(x)} = \frac{f_{X}(x) \mathbbm{1} \{X \in A\}}{P(X \in A)}
\end{equation*}

\section{Vectores Aleatorios}
$\mathbb{X}  = (X_{1},X_{2},X_{3},...,X_{n})$ es un vector aleatorio de dimension $n$ si para cada $j = 1,...,n$; $X_{j}: \Omega \to \mathbb{R}$ es una
V.A.
\subsection{Funcion de distribucion de un vector aleatorio}
Sea $\mathbb{X}$ un vector aleatorio de dimension $n$, definimos la funcion de distribucion de $\mathbb{X}$ como:
\begin{equation*}
    F_{\mathbb{X}}(\bar{x}) = P(X_{1} \leq x_{1},X_{2} \leq x_{2}, X_{3} \leq x_{3},..., X_{n} \leq x_{n})
\end{equation*}

\subsection{Propiedades del vector aleatorio ($\mathbb{X} = (X,Y)$)}
\begin{enumerate}
    \item $\lim_{x,y \to \infty} F_{\mathbb{X}}(x) =1$,
\end{enumerate}

\end{document}
